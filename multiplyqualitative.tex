%!TEX TS-program = xelatex 
%!TEX encoding = UTF-8 Unicode
\documentclass[12pt]{article} 

% Definitions
\newcommand\mykeywords{color, selectionism, Shoemaker} 
\newcommand\myauthor{Mark Eli Kalderon} 
\newcommand\mytitle{The Multiply Qualitative}

% Packages
\usepackage{geometry} \geometry{a4paper} 
\usepackage{pdfsync} 
\usepackage{url} 
\usepackage{txfonts}
\usepackage{color}
\definecolor{gray}{rgb}{0.459,0.438,0.471}
\usepackage{epigraph}

% XeTeX
\usepackage[cm-default]{fontspec}
\usepackage{xltxtra,xunicode}
\defaultfontfeatures{Scale=MatchLowercase,Mapping=tex-text}
\setmainfont{Hoefler Text}
\setsansfont{Gill Sans}
\setmonofont{Inconsolata}

% Bibliography
\usepackage[round]{natbib}

% Version Control Information
\immediate\write18{sh ./vc}
\input{vc}

% Section formatting
\usepackage[]{titlesec} 
\titleformat{\section}[hang]{\fontsize{14}{14}\scshape}{\S\,{\thesection}}{.5em}{}{}

% Headers
\usepackage{fancyhdr}
\pagestyle{fancy}
\pagenumbering{arabic}
\lhead{\thepage}
\chead{}
\rhead{\itshape{\nouppercase{\leftmark}}}
\lfoot{\tiny{Revision: \VCRevision}}
\cfoot{\tiny{Author: \VCAuthor}}
\rfoot{\tiny{Date: \VCDateISO}}

% Title Information
\title{\mytitle\thanks{Thanks to Keith Allen, David R. Hilbert, Carrie Jenkins, Guy Longworth, MGF Martin, Daniel Nolan, Sydney Shoemaker, Maja Spener, Scott Sturgeon, and Charles Travis for their help and encouragement. Thanks also to audiences at Birkbeck College and the University of Nottingham where a version of this material was presented.}} 

\author{\myauthor}
% \date{}

% PDF Stuff
\usepackage[plainpages=false, pdfpagelabels, bookmarksnumbered, backref, pdftitle={\mytitle}, pagebackref, pdfauthor={\myauthor}, pdfkeywords={\mykeywords}, xetex, colorlinks=true, citecolor=gray, linkcolor=gray, urlcolor=gray]{hyperref} 

%%% BEGIN DOCUMENT
\begin{document}

\maketitle

\begin{abstract}
	Shoemaker argues that one could not hold both that the qualitative character of color experience is inherited from the qualitative character of the experienced color \emph{and} that there are faultless forms of variation in color perception. In this paper, I explain what is meant by inheritance and discuss in detail the problematic cases of perceptual variation. In so doing I argue that these claims are in fact consistent, and that the appearance to the contrary is due to an optional and controversial conception of experience that should be rejected.
\end{abstract}


%	\tableofcontents
\setlength{\parindent}{1em}

% \vskip 2em \hrule height 0.4pt \vskip 2em

\epigraph{God is day and night, winter and summer, war and peace, surfeit and hunger; but he takes various shapes, just as fire, when it is mingled with spices, is named according to the savor of each.}{\textsc{Heraclitus}}

\section{Introduction}\label{sec:introduction} 

% (fold)
When Norm perceives a red tomato, there is a way it is like for Norm to undergo that experience. Norm's experience of the red tomato has a distinctive phenomenal character. What is the relationship between the phenomenal character of Norm's experience and the perceived color? One na\"ive thought is this---the phenomenal character of color experience is determined by the qualitative character of the perceived color. When Norm perceives a red tomato, the qualitative character of his color experience is determined by the qualitative character of the color manifest in his experience of the tomato. As \citet[189]{Campbell:1997dq} puts it, ``the qualitative character of the color experience is inherited from the qualitative character of the color.''

\cite{Shoemaker:wk}, however, in a paper critical of \cite{Hilbert:2000on}, argues that the na\"ive commitment to chromatic inheritance cannot be sustained. I will take up Shoemaker's criticisms with an eye to vindicating the inheritance thesis and thus partially vindicating the na\"ive conception of color and its relation to color experience.

% section introduction (end)
\section{Inheritance}\label{sec:inheritance} 

% (fold)
How are we to understand Campbell's metaphor of inheritance? 

Campbell's metaphor embodies a claim about color and color phenomenology. It is, however, an instance of a more general claim---that the phenomenal character of experience is inherited from the objects, qualities, and relations present in experience. 

At a minimum, this involves the following claim: 
\begin{quote}
	Necessarily, a difference in the objects, qualities, and relations present in experience suffices for a difference in the phenomenal character of experience. 
\end{quote}

As presently formulated, the claim is noncommittal as to the nature of the objects, qualities, and relations present in experience. Thus, for example, sense-datum theorists such as \citet{Price:1932fk} maintain that reflection on the problems of illusion and hallucination reveal that the objects present in perceptual experience are nonmaterial and that the qualities and relations present in experience are qualities and relations of these nonmaterial objects. As opposed to this, representationalists and na\"ive realists maintain that the objects present in perceptual experience can be material and that the qualities and relations present in experience can be qualities and relations of these material objects. On this, I side with the representationalists and na\"ive realists. Though I provide no argument for this claim, I will assume the following for the purposes of this paper: 
\begin{quote}
	Objects, qualities, and relations of the material environment can be present in a subject's perceptual experience of that environment. 
\end{quote}
More specifically, and controversially, I will assume that: 
\begin{quote}
	Colors are among the mind-independent qualities of the material environment that can be present in a subject's perceptual experience. 
\end{quote}

The claim is also noncommittal as to the nature of perceptual presentation. Representationalists maintain that the sensible qualities present in experience determine at least some of the phenomenal properties of that experience. Moreover, they maintain that the sensible qualities present in experience just are the sensible qualities that that experience represents. In so doing, they endorse a substantive and controversial claim about perceptual presentation---that perceptual presentation just is perceptual representation. As opposed to this, sense-datum theorists and na\"ive realists maintain that perceptual presentation is nonrepresentational. For the purposes of this paper, I will be neutral about the representational character of perceptual presentation.

Moreover, this is not yet to endorse the converse claim:
\begin{quote}
    Necessarily, a difference in the phenomenal character of experience suffices for a difference in the objects, qualities, and relations present in experience. 
\end{quote}
The phenomenal properties of experience may, after all, be due, in part, to the objects and relations present in that experience. Even if understood inclusively in this way---that, necessarily, a difference in the phenomenal properties of experience suffice for a difference in the objects, qualities, or relations present in that experience, the claim may still be intelligibly doubted. Perhaps the \emph{way} something is presented in experience, as well as what's presented, can make for a phenomenal difference. Thus, for example, \citet{Martin:2002cr} argues that the phenomenal difference between the perception of a sensible quality and the sensory imagining of that quality is due to the way the sensible quality is presented in perception and sensory imagination, respectively. Extending this to the case of color, the phenomenal difference between perceiving a color and imagining a color is due to the different ways in which the color is presented in perception and visual imagination, respectively. Perhaps, then, phenomenology is not exhaustively presentational. But where it is, a difference in the relevant phenomenal properties suffices for a difference in the sensible objects, qualities, and relations present in experience.

Finally, to claim that a phenomenal property is inherited from what is presented in experience is to claim more than certain phenomenal properties covary with something present in experience, even if the covariation is counterfactual. It involves as well an explanatory claim---an experience has the relevant phenomenal property \emph{because} of what is present in experience. This is implicit in the modal implications of Campbell's metaphor of ``inheritance''. To claim that the qualitative character of color experience is inherited from the qualitative character of the presented color is to claim that the qualitative character of the experience \emph{depends on} and \emph{derives from} the qualitative character of the presented color. 

While the explanatory claim entails that the relevant aspect of phenomenal character covaries with something present in experience, the converse entailment fails. Thus, for example, \citet{Chalmers:2006kx} accepts that the phenomenal properties of experience covaries with what's present in experience (where perceptual presentation is understood representationally), but maintains that experience represents what it does because of its phenomenal properties: 
\begin{quote}
	A \emph{phenomenal content} of a perceptual experience is a representational content that is determined by the experience's phenomenal character. \citep[50]{Chalmers:2006kx} 
\end{quote}
Though Chalmers is not as careful as he might be to distinguish the explanatory and covariation claims as he continues: 
\begin{quote}
	More precisely: a representational content $c$ of perceptual experience \textsc{e} is a phenomenal content if and only if necessarily, any experience with the phenomenal character of \textsc{e} has representational content $c$. \citep[50]{Chalmers:2006kx} 
\end{quote}
This is merely a claim of necessary covariation that lacks the explanatory asymmetry entailed by talk of determination. Indeed it is a claim of necessary covariation accepted even by those who accept the converse order of explanation---that experience has the phenomenal properties that it has because of its representational content. For the same reason, I believe that \citet{Byrne:2001dq} is wrong to formulate representationalism as a supervenience thesis---it is rather an explanatory claim with the supervenience thesis as a consequence. (See \citealp{Hilbert:2000on}, for a variant argument for this claim, and see \citealp{Martin:2002cr}, for further relevant discussion.)

% Subject to the above qualifications, let the \emph{chromatic inheritance thesis} be the following claim: 
% \begin{quote}
%   Color experience inherits its phenomenal character from the qualitative character of the color present in it just in case:
%   \begin{enumerate}
%       \item Necessarily, a difference in the color present in experience suffices for a difference in the phenomenal character of experience. 
%       \item Necessarily, a difference in the phenomenal character of experience suffices for a difference in the color present in experience. 
%       \item The qualitative character of the color present in experience determines its phenomenal character---an experience has the color phenomenology it has \emph{because of} the qualitative character of the color present in it.
%   \end{enumerate}
% \end{quote}

Whether and to what extent chromatic inheritance can be sustained depends, in part, on the cogency of Shoemaker's criticisms. The case against the necessary covariation between the qualitative character of the presented color and the phenomenal character of the experience that presents it will be considered in section~\ref{sec:the_selection_problem}. In section~\ref{sec:the_multiply_qualitative}, we will see how the necessary covariation might, however, be sustained given a metaphysical hypothesis about the colors---that the colors have multiple qualitative natures. In sections~\ref{sec:inhertance_and_chromatic_causation} and~\ref{sec:intrinsic_and_extrinsic_powers}, we will see that Shoemaker argues that the necessary covariation can only be sustained in this way at the expense of the explanatory asymmetry. According to Shoemaker, then, chromatic inheritance ultimately fails because the explanatory asymmetry involved in talk of `inheritance' cannot be sustained. 

% section inheritance (end)
\section{Selectionism}\label{sec:selectionism} 

% (fold)
Central to the account of \cite{Hilbert:2000on} is the metaphor of selection. The metaphor of selection is meant to provide an interpretation of the dependency of color perception on the visual sensibility of a perceiver consistent with the colors being mind-independent features of the material environment. If the visual sensibility of a perceiver selects which features of the material environment are perceptually available, then the perceptual availability of the colors will depend on the visual sensibility of a perceiver. However, there is nothing contradictory, or otherwise internally incoherent, about the visual system of the perceiver partly determining the perceptual availability of mind-independent qualities. \citep[See][39--53.]{Price:1932fk}

Begin with an abundant conception of properties. On such a conception, there are indefinitely many regularities that obtain in a perceiver's environment. Some of these regularities are more natural than others---those grounded in the sparse properties of the material environment will be more natural than those that are not. Not all of the indefinitely many regularities present in the material environment are manifest in the perceiver's experience of the scene. The visual sensibility of a perceiver selects which of these regularities are perceptually available to the perceiver. \citet{Shoemaker:wk} sympathetically and insightfully characterizes selectionism as follows: 
\begin{quote}
	For any ordered set of properties we can define a similarity relation such that the degree of similarity of two properties in the set is determined by how close they are to each other in that ordering. Perhaps most of these should count only as relations of ``quasi-similarity.'' But what determines which of these relations count as ``real'' or ``genuine'' similarity relations? A first step towards an answer is to say that such a relation is a genuine similarity relation if it makes properties similar to the extent that their instantiation bestows similar causal powers. But what sorts of causal powers are relevant will vary depending on our interests. In the case of sensible properties of things, the relevant powers include the powers to affect the experiences of perceivers; and in the case of the so-called ``secondary qualities'' these are close to being the only powers that are relevant. Powers to affect experiences will be grounded in powers to affect the physical states of perceptual systems. And given that a perceptual system realizes a repertoire of perceptual experiences standing in certain similarity relations, there is an obvious sense in which its physical nature determines what properties bestow the powers to produce in the possessor of the system experiences belonging to that repertoire, and what relations among these properties bestow similarities with respect to these powers. In this sense the nature of the perceptual system ``selects'' what properties are to count as sensible properties, and what relations among them are to count as similarities with respect to these properties. 
\end{quote}
I accept this characterization with the exception of one minor infelicity. It is not the similarities among perceptual experiences that determines which relations are perceived as color similarities; rather, it is the ordering on potential states of the visual system of the perceiver---states that perhaps \emph{constitute}, at least in part, perceptual experiences---that determines which relations are perceived as color similarities. (The significance of this qualification will emerge in section~\ref{sec:intrinsic_and_extrinsic_powers}.)

The selective activity of the visual sensibility is consistent with the selected similarities supervening on mind-independent colors. Price offers the following analogy: 
\begin{quote}
	If I am to select a bun from the counter my hand must be there to pick it up. If I move my hand to the left I pick up bun No. 1, if to the right, bun No. 2. But the bun which I do pick up is in no way dependent upon my hand for its existence, nor my hand upon the bun. Hand plus bun do not form an organic whole, and either could exist without the other. Still less can we say that the hand creates the bun. \citep[40]{Price:1932fk} 
\end{quote}
The selective activity of the visual sensibility does not determine color similarities; rather, it determines the perceptual availability of these similarities and, hence, the perceptual availability of the colors. Of course, the selected relations and the colors they supervene on will reflect the nature of the perceiver's visual sensibility. Colors are thus anthropocentric in something like Wiggins' \citeyearpar{Wiggins:1987ta}, and Hilbert's \citeyearpar{Hilbert:1987jq}, sense of the term. But being anthropocentric makes colors neither less real nor less mind-independent. The selected family of colors might not be very natural (though natural enough for their instances to be among the causal antecedents of color perception), the selected family of colors might only be perceptually available in certain circumstances of perception or to certain perceivers, but the colors could be mind-independent qualities of material objects for all that.

Selectionism, as presently understood, has an important consequence. If the visual sensibility of the perceiver selects which of the indefinitely many regularities of the material environment are perceptually available to him, then perception is \emph{partial}. Not only is perception partial in the sense that there are properties of an object not perceptually available (objects may have unobservable aspects), not only is perception partial in the sense that some sensible qualities of an object may be occluded from view (the backs of objects are colored as well), but perception is also partial in the sense there are perceptually available properties of an object that are not determined by a given perception. If there is more to the sensible qualities of an object than is manifest in a given perception, then not only might different sensible qualities of an object be perceptually available only in different circumstances of perception, but different sensible qualities of an object might be perceptually available only to different perceivers. The partiality of perception has recently been defended by \citet{Hilbert:1987jq}, but it has ancient roots as well---arguably, Heraclitus is an advocate \citep[see][]{Burnyeat:1979mv,Kalderon:2006tg}.

% section selectionism (end)
\section{The Selection Problem}\label{sec:the_selection_problem}

% (fold)
The visual sensibility's selection of the colors is not exclusive. In this way it differs from the selection of teams by opposing captains. One captain's selection of a player as a member of his team excludes the other captain's selection of that player as a member of the other team. Once a player is selected, that player is not available to be selected by the opposing captain. But the selection of a property as a member of a family of colors perceptually available to one kind of perceiver does not exclude the selection of that property as a member of a distinct family of colors perceptually available to a distinct kind of perceiver. The properties selected to be the colors by distinct perceivers might not themselves be distinct.

There are two ways in which the selected properties can fail to be distinct. Let the extension of a property be the plurality of objects that instantiate it: 
\begin{itemize}
	\item The selected properties might be coinstantiated in which case their extensions \emph{overlap}. 
	\item The selected properties might be identical in which case their extensions necessarily \emph{coincide}. 
\end{itemize}
Cases of overlap are not only possible, but are plausibly actual---cases of veridical interspecies perceptual variation are plausibly of this form. Cases of coincidence are improbable, but seem at least logically possible given selectionism. 

Whereas cases of overlap are metaphysically unproblematic, cases of coincidence are metaphysically problematic---or so Shoemaker contends: 
\begin{quote}
	But suppose that subjects $S_1$ and $S_2$ have differently structured color quality spaces, but that one of $S_1$'s colors, call it $c_1$, has as surface color realizers the same set of reflectances as one of $S_2$'s colors, call it $c_2$. More generally, suppose that $c_1$'s total set of realizers, those for colored lights and transparent or translucent solids as well as surface colors, is identical with $c_2$'s total set of realizers. Nothing in the selection account rules this out, and it seems perfectly conceivable. If the possible realizers of $c_1$ are the same as those of $c_2$, it is hard to resist the conclusion that $c_1$ and $c_2$ are the same color. But if they are the same color, then perceptual systems with differently structured experiences spaces can ``select'' the same property in the world as one of the colors while selecting different similarity relations between it and other colors. Assuming that this would not involve systematic misperception on the part of the possessors of one of the perceptual systems, and there is no reason to think it would, this contradicts the view of Hilbert and Kalderon that the colors are individuated by their similarity relations. And if, as they claim, the phenomenal character of color experiences is determined by what color similarities they represent, it would seem that it gives us a case in which veridical experiences of the same color, in the same viewing conditions, differ in phenomenal character. Given this possibility, it certainly does not seem that the phenomenal character of color experiences can be simply inherited from the nature of the colors they represent. \citep{Shoemaker:wk} 
\end{quote}
We can reconstruct Shoemaker's argument as follows: Internal relations of similarity and difference can be represented by external relations of distance in a space. Let a \emph{color experience space} be a representation of the phenomenal similarities and differences among actual and potential color experiences. Let a \emph{color property space} be a representation of similarities and differences among a family of color properties. Let Norm and Norma be perceivers with differently structured color experience spaces. Given their differently structured color experience spaces, Norm and Norma's visual sensibilities select families of colors that constitute differently structured color property spaces. Thus, for example, Norm's visual sensibility determines a color experience space, an ordering on actual and potential color experiences. Instances of certain properties in Norm's environment tend to cause, in certain conditions, Norm to have one of these color experiences. Given these causal powers, the properties in Norm's environment are themselves ordered in a way that mirrors the ordering of potential color experiences. And since Norma has a differently structured color experience space, the properties of the material environment whose instances tend to cause, in certain circumstances, Norma to have a certain kind of experience are ordered in a way that mirrors the ordering of Norma's potential color experiences and so participate in a distinct color property space to the color properties perceptually available to Norm. Let $c$ be a property selected to be a color by Norm and Norma's visual sensibilities. Norm and Norma's experiences of $c$ are phenomenally different---what it is like for Norm to perceive $c$ in a given circumstance of perception is different for what it is like for Norma to perceive $c$ in the same circumstances of perception. But then, the chromatic inheritance thesis would be false---there would be a difference in the phenomenal properties of color experience without a difference in the color present in experience.

Not only would there be a difference in the phenomenal properties of color experience without a difference in the color present in experience, but the converse claim apparently fails as well---there could be a difference in the color present in experience without a difference in the phenomenal properties of color experience. Suppose that Norm and Norma have identically structured color experience spaces. For every potential color experience of Norm's, there would be a potential color experience of Norma's that is phenomenally identical, and for every potential color experience of Norma's, there would be a potential color experience of Norm's that is phenomenally identical. But suppose that the color experience spaces of Norm and Norma are anchored to different features of the material environment---``Their visual systems differ in such a way that they select somewhat different properties as the colors, and somewhat different relations as the relations of color similarity and difference'' \citep[p.\ 263]{Shoemaker:wk}. It is possible that there be two properties, $c_1$ and $c_2$, such that Norm's experience of $c_1$ is phenomenally just like Norma's experience of $c_2$ in which case a difference in the color present in experience would be insufficient for a difference in the phenomenal properties.

It would seem, then, that the color present in color experience would be neither necessary nor sufficient for the phenomenal properties of that experience. If the color present in experience is neither necessary nor sufficient for the phenomenal character of that experience, then the phenomenal character of color experience is not inherited from the color present in that experience.


% (end)

\section{The Multiply Qualitative}\label{sec:the_multiply_qualitative} % (fold)

Should the chromatic inheritance thesis be rejected then? Perhaps not: 
\begin{quote}
	Suppose that the different perceptual systems ``select'' the same properties as maximally determinate colors (as it might be, select the same set of reflectances to be the maximally determinate surface colors), but differ in the similarity relations they select in such a way that they differ in the way they group these determinate colors into color determinables or color categories. The difference in the phenomenal character of the experiences that the possessors of these perceptual systems have of one of these determinates could then be a matter of their representing the possessors of that property as having different determinable properties. Or, to put it slightly differently, one of them perceives the determinate as a determinate of one determinable, and the other perceives it as a determinate of a different determinable, and it is this difference in representational content that accounts for the difference in phenomenal character of their experience. \citep[266]{Shoemaker:wk} 
\end{quote}

As Shoemaker observes, it is arguable that this actually happens. Shoemaker suggests that the intersubjective variation in the spectral location of the unique hues might be such a case. If asked to adjust a green light such that it is not at all bluish and not at all yellowish, normal perceivers will consistently do so within 3nm. In contrast, \emph{inter}subjective variation in the spectral location of the unique hues is remarkably wide. The spectral location of the unique hues varies among normal perceivers by as much as ten percent of the visible spectrum. Thus, something that appears bluish green to one normal perceiver can appear unique green to another normal perceiver and yellowish green to a third \citep[see][]{Leon-M.-Hurvich:1968fu}. Suppose that an object looks unique green to Norm and yellowish green to Norma in the same circumstances of perception. Shoemaker's suggestion is that Norm and Norma are seeing the same determinate shade of color but are perceiving it to be a determinate shade of different determinables. This is controversial, however, \citep[see][for some alternatives]{Byrne:2007qy,Cohen:2006fj,Cohen:2007kx,Kalderon:2006tg,Tye:2006lr,Tye:2006yq, Triplett:2007uq}. 

Certain forms of red--green color blindness, such as mild forms of deuteranomaly, constitute a better case, I think. Deuteranomaly is the result of a mutation in the medium wavelength pigment resulting in a reduction in sensitivity to green portion of the spectrum. Approximately six percent of the male population are subject to this mutation (though some estimates are higher). Suppose that a standard Ishihara test reveals Norm to be a deuteranomolous perceiver. Suppose, however, that Norma's color vision is not ``deficient'' in this way. (The scare-quotes are apt since color blind perceivers can outperform normal color perceivers in certain perceptual tasks. Thus the military has discovered that color blind perceivers are less prone to be taken in by camouflage.) In certain circumstances of perception, Norm is prone to take a green thing to be red. It is not the case that Norm cannot see the difference between red and green. Broackes, himself a deuteranomolous perceiver, claims: 
\begin{quote}
	\ldots I do not have a single kind of perception from red, green, and grey things in general. I have no difficulty in seeing the red of a post-box, or the green of the grass, and my identification of their colour is not due to knowing already what kind of thing I am looking at. (I am equally good on large blobs of paint.) \cite[p.\ 216]{Broackes:1997pa} 
\end{quote}
It is plausible, then, that Norm sees the same determinate shade of green as Norma. It is just that their visual systems apply different color categories to this shade such that they see it as falling under different color determinables with the result that, in certain circumstances, Norm is prone to confuse it with a certain shade of red. %If Norma apprises Norm of his mistake, or if Norm views the green thing in different conditions of illumination, it is plausible that Norm can come to see the green thing \emph{as} green. 

If there were veridical cases of this kind of perceptual variation, if a determinate color could fall under different determinables thus allowing it to bear different similarity relations to different properties and so participate in distinct color property spaces, then colors would have multiple qualitative natures. A single determinate color would have a qualitative nature perceptually available to a certain kind of perceiver and a different qualitative nature perceptually available to a different kind of perceiver. 

Parallels with the phenomena of color constancy and metamerism, understood in terms of a presentational phenomenology, provide some support for this hypothesis. Color constancy is the capacity of color to appear to persist, unaltered, through changes in its appearance across a broad range of scenes and conditions of illumination. Thus a tomato can look to be a particular shade of red, and the same shade of red, in fluorescent lighting, in noon daylight, and on an overcast afternoon though it appears differently in each of these conditions of illumination. Metamerism is the capacity of different colors to match in color appearance in some scenes and conditions of illumination. Thus two garments can match in color appearance when viewed in a store and yet fail to match in color appearance when viewed in sunlight. If the phenomenology of color constancy is presentational, then the different appearances the color presents are different qualitative aspects of the color perceptually available to the perceiver in different circumstances of perception \citep[see][for a defense of this]{Hilbert:2006uq,Kalderon:2006fk}. And if the phenomenology of metamerism is presentational, then different colors share a qualitative aspect perceptually available to the perceiver in the circumstance of perception \citep[see][for a defense of this]{Kalderon:2006fk}.

What, then, are the parallels?

First, when Norm and Norma have phenomenally different experiences of an identical color $c$, there is, nevertheless, different things present in their respective experiences that can explain this phenomenal difference---$c$ presents different qualitative aspects of its nature to Norm and Norma in the circumstances of perception. This structurally parallels the case of color constancy---the way in which a color appears to persist, unaltered, through the changes in its appearance across a range of scenes and conditions of illumination. That is an \emph{intra}subjective case of different qualitative aspects of a color being perceptually available to a perceiver in different circumstances of perception. The present case is an \emph{inter}subjective case of different qualitative aspects of a color being perceptually available to different perceivers in the same circumstance of perception. If perception provides only a partial perspective on the sensory aspects of the material environment, as a Heraclitean epistemology would have it, not only is it possible that different aspects of a color's qualitative nature are perceptually available to a perceiver in different circumstances of perception, but it is also possible that different aspects of a color's qualitative nature are perceptually available to different perceivers in the same circumstance of perception.

Second, when Norm and Norma have phenomenally identical experiences of the distinct colors, $c_1$ and $c_2$, there is, nevertheless, something commonly present in their respective experiences that can explain this phenomenal identity---$c_1$ and $c_2$ present the same qualitative aspect to Norm and Norma in the circumstance of perception. This structurally parallels the case of metamerism---the way in which two colors can match in color appearance in certain conditions of illumination. That is an \emph{intra}subjective case of different colors sharing a qualitative aspect perceptually available to a perceiver in the circumstance of perception. The present case is an \emph{inter}subjective case of different colors sharing a qualitative aspect perceptually available to different perceivers in the same circumstance of perception. If perception provides only a partial perspective on the sensory aspects of the material environment, as a Heraclitean epistemology would have it, not only is it possible that different colors share a qualitative aspect perceptually available to a perceiver in the circumstance of perception, but it is also possible that different colors share a qualitative aspect perceptually available to different perceivers in the same circumstance of perception.

If colors have a multiple qualitative natures, then selectionism is, after all, consistent with chromatic inheritance.

% section the_multiply_qualitative (end)

\section{Chromatic Inheritance and Causation}\label{sec:inhertance_and_chromatic_causation}

% (fold)
\citet{Shoemaker:2006vn} is sympathetic to the idea that properties can have multiple qualitative natures. However, he doubts whether the chromatic inheritance thesis can be reconciled with veridical perceptual variation by positing colors with multiple qualitative characters: 
\begin{quote}
	If indeed standard representationalism can be made compatible with the possibility of spectrum inversion without misperception, that removes my main objection to it. But I think that is questionable whether allowing a color property to have multiple qualitative characters, in the way required if we are to reconcile the ``inheritance thesis'' with the relativity of color similarity, is really compatible with standard representationalism. \ldots Suppose that a given property occupies different positions in the color property space of creatures \textsc{a} and \textsc{b}, so their experience of it (in the same viewing conditions) are phenomenally different. But suppose further that the proximate effects of the instantiation of the property on the visual systems of \textsc{a} and \textsc{b} are the same---the difference is due to the fact that the initial input, which is the same in both, is processed in different ways in the two systems. It seems plausible to take a qualitative character of a color to be individuated by a subset of the causal features of the property, namely, those involved when the instantiation of the property cause a color experience in a creature with a certain sort of perceptual system. \ldots But in the case imagined, it will be one and the same set of causal features of the color property that is the external source of the phenomenally different color experiences its instantiation causes. And it hardly seems that we can say that the experience inherits different phenomenal characters from the same qualitative character of the property. \citep[p.\ 269]{Shoemaker:wk} 
\end{quote}

We can reconstruct Shoemaker's argument as follows.

Let Norm and Norma be perceivers with differently structured color experience spaces. Given their differently structured color experience spaces, Norm and Norma's visual sensibilities select families of colors that constitute differently structured color property spaces. Let $c$ be a property selected to be a color by Norm and Norma's visual sensibilities. Norm and Norma's experiences of $c$ are qualitatively different---what it is like for Norm to perceive $c$ in a given circumstance of perception is different for what it is like for Norma to perceive $c$ in the same circumstances of perception. This might be reconciled with the inheritance thesis, however, if $c$ had multiple qualitative characters. When Norm perceives $c$, he perceives what $c$ is like, but not in all respects. $c$'s qualitative nature is only partially manifest in Norm's perception of it---there are qualitative aspects to $c$'s nature that are not perceptually available to him, but are perceptually available to Norma. So the qualitative difference between their color experiences is explained in terms of the different qualitative natures of $c$ manifest in their perceptions of it. $c$ will thus belong to distinct if overlapping color property spaces.

However, there is a problem with this putative reconciliation. Suppose that the qualitative nature of a color is a subset of its causal powers, namely those involved in the production of color experiences. But suppose further that the proximate effects of $c$'s instantiation on Norm and Norma are the same---the fact that $c$'s instantiation elicits qualitatively different color experiences is entirely due to further processing by their respective visual systems. Given the sameness of proximate effects, the causal powers involved in $c$'s instantiation causing Norm and Norma's color experiences are themselves the same. If the qualitative nature of a color really is a subset of its causal powers involved in the production of color experiences, then $c$'s qualitative nature isn't multiple, it is unitary. There could be a phenomenal difference between Norma and Norma's color experience without a difference in what's presented in their respective experiences thus contradicting chromatic inheritance.

In response to this argument, one might query the background metaphysics of properties, a metaphysics according to which properties quite generally are causal powers. Unfortunately, this won't help. Suppose that $c$'s qualitative nature is something over and above the subset of causal powers involved in the production of experiences of it. Let $q_1$ be the qualitative nature of $c$ that Norm perceives and let $q_2$ be the qualitative nature of $c$ that Norma perceives. Given the sameness of proximate effects, the causal powers involved in $c$'s instantiation causing Norm and Norma's color experiences are the same. But then it would seem that $q_2$ is just as causally responsible for the qualitative character of Norm's experience as $q_1$; and $q_1$ is just as causally responsible for the qualitative character of Norma's experience as $q_2$. It remains hard to understand how the qualitative character of a color experience is inherited from the qualitative character of the perceived color. (See \citealp{Johnston:2005dq}, for a similar argument.)

Perhaps this difficulty is due to a substantive assumption about the metaphysics of color---that colors are monadic properties of objects in which they inhere. So understood, the color of a tomato depends on how the tomato is in and of itself and apart from any other thing. Suppose, however, that colors were not monadic but relational---perhaps they are determined by the relations that obtain between the object, perceiver, and circumstances of perception. This would be a kind of Protagorean relativism. The Protagorean need not deny that the qualitative nature of a color is a subset of its causal powers involved in the production of color experiences. The Protagorean need only deny that the relevant subset of causal powers are antecedent to the proximate effects on color perceivers. If color is relational, if colors are determined by relations that obtain between objects, perceivers, and circumstances of perception, then among the causal powers would plausibly be those involved in the further visual processing. If distinct visual processing is required to produce Norm and Norma's qualitatively distinct color experiences, then Norm and Norma would be perceiving distinct relational qualities with distinct qualitative natures. The Protagorean response works, if it works at all, by reducing an apparent case of coincidence to the less metaphysically problematic case of overlap. (\citealp{Johnston:2005dq}, argues for color relativism on these grounds; color relativism has also recently been defended by \citealp{Cohen:vl,McLaughlin:2003cr}.)

It is an open question whether the Protagorean response can be made to work. \citet{Shoemaker:wk} doubts whether it can. However, whether or not the Protagorean response can be made to work is irrelevant, for realtivism is unnecessary to resolve the problem and so must be motivated on other grounds.

Perhaps the real difficulty is not posed by the identification of the qualitative natures of the colors with subsets of causal powers involved in the production of color experience, nor by any assumption about the extension of the relevant subset of causal powers, but by the assumption that the proximate effects of $c$'s instantiation on Norm and Norma are the same. Allowing for a reasonable amount of vagueness about what exactly counts as proximate, it is at least arguable that, over and important range of actual cases of shifted spectra, the proximate effects of a color on subjects with qualitatively distinct color experiences are themselves distinct. Specifically, in many such cases, the phenomenal difference is the effect of different patterns of retinal stimulation. Thus the phenomenal difference between the experience of a normal color perceiver and a deuteranomolous perceiver is due to a mutation in the medium wavelength pigment in the latter with the result that, in the circumstance of perception, the proximate effects on these perceivers will differ---specifically, the peak sensitives of their cones will differ. Not only will a difference in the peak sensitivities of the cones result in variation in color vision, but so will varying the shape of the sensitivity curves. And the intersubjective variation in the spectral location of the unique hues is similarly due to a difference in the retinal effect of the visual stimulus. So, over an important range of actual cases of shifted spectra, the phenomenal difference in color experience is due to a difference in the proximate effects on the perceivers, on a reasonable understanding of that notion.

The relevance of this observation might be questioned. Recall that the problematic cases for selectionism are cases of coincidence---cases where the phenomenally different color experiences present the same color and so their extensions necessarily coincide. Whereas cases of overlap are plausibly actual, cases of coincidence are, at best, hypothetical. But if the problematic cases are merely hypothetical, how does the fact that in actual cases of shifted spectra the proximate effects differ bear on whether in hypothetical cases the proximate effects would differ? Couldn't we simply imagine that, in the relevant hypothetical case, the phenomenal difference is due to further visual processing?

One might wonder what exactly are we being asked to imagine. The relevant case is so far underdescribed---we lack an explanation of the source of perceptual variation. While we can clearly conceive that the phenomenal difference is due to further visual processing, without a further explanation of the source of the perceptual variation, we cannot \emph{distinctly} conceive this. And if we cannot clearly and distinctly conceive this, we so far lack a reason to believe this to be genuinely possible. The worry, while genuine, is too weak, however. While we may so far lack a reason to believe that it is possible that the phenomenal difference is due to further visual processing, this is not yet to claim that there could be no such reason. Further argument is required.

% (end)
\section{Intrinsic and Extrinsic Powers}\label{sec:intrinsic_and_extrinsic_powers} 

% (fold)
The real difficulty with the assumption that the proximate effects of $c$'s instantiation on Norm and Norma are the same lies with a tacit and optional conception of perceptual experience in terms of which the assumption is understood. 

To bring this out, consider how Shoemaker thinks that the necessary covariation between color phenomenology and the presented color can be preserved but only at the expense of the explanatory asymmetry crucial to talk of ``inheritance'': 
\begin{quote}
	Now in the present example, the color has both the power to produce one sort of experience in the likes of [Norm] and the power to produce another sort of experience in the likes of [Norma], and while these powers are grounded in the same causal features of the color there is a sense in which they are different---one is a power to produce one effect, and the other is the power to produce a different effect. So we might preserve the necessary correspondence by taking these different powers to be the different qualitative characters that are presented by the color to the different observers. But given that the causal features that ground one of these powers are the same as those that ground the other, and that the powers are different because of the different phenomenal characters that experiences have when they are exercised, it would only be a very Pickwickean sense that the phenomenal characters of the experiences could be said to be ``inherited from'' qualitative characters of the colors. \citep[476, n. 8]{Shoemaker:2006vn} 
\end{quote}

The qualitative nature of a color is conceived to be an \emph{extrinsic} causal power. (Shoemaker cites Robert Boyle's example of a key's power to open a door as an example of an extrinsic causal power---it is an extrinsic causal power since, without altering the key, we can deprive it of that power by changing the lock.) The qualitative nature of a color is a subset of its causal powers, namely those involved in the production of color experiences. By hypothesis, the color $c$ presents distinct qualitative aspects, $q_1$ and $q_2$, to Norm and Norma, respectively. If the proximate effects on Norm and Norma are the same, then the causal features of $c$ that ground $q_1$ and $q_2$ are the same. Thus if $q_1$ and $q_2$ are genuinely distinct causal powers, they could not be \emph{intrinsic} causal powers. What distinguishes them is the phenomenally distinct color experiences they elicit in Norm and Norma. But this is inconsistent with the explanatory asymmetry involved in talk of inheritance. Specifically, if a color experience has the phenomenal properties that it does \emph{because} of the qualitative aspect of the color it presents, the qualitative aspect must be individuated independently of the phenomenal experience it elicits. But the distinct qualitative aspects of $c$ are individuated, in part, by the phenomenally distinct color experiences they elicit. The phenomenal character of color experience could not depend on and derive from the qualitative aspect of the color present in that experience.

To reject this explanatory asymmetry is not yet to accept the converse explanatory asymmetry---that the qualitative character of a color depends on and derives from the phenomenal character of the color experience it elicits. Indeed, insofar as Shoemaker is a representationalist, albeit of a nonstandard sort, he must reject the converse explanatory asymmetry as well. He must maintain, instead, that the phenomenal character of color experience and the qualitative character of the color present in color experience are \emph{interdependent}, or, if this does not come to the same thing, that they are \emph{codetermined}.

If what distinguishes $q_1$ and $q_2$ as distinct causal powers is the phenomenally distinct experiences they elicit, phenomenal experience must be conceived in a certain way---as a way of being affected. So conceived, a phenomenal experience is a conscious modification of a subject. Placing an object a certain distance from another does not modify that object, only its location---though, of course, changing the distance among its parts will modify an object. Thus moulding a lump of clay into triangle modifies that lump of clay. If experience is a way of being affected, then experience is a modification of the perceiving subject in the way that being triangular is a modification of the clay. But whereas experience is a \emph{conscious} modification, being triangular is not. Compare Reid's characterization of our experience of ``secondary qualities'': 
\begin{quote}
    \ldots\ our senses give us a direct and distinct notion of the primary qualities, and inform us what they are in themselves: but of the secondary qualities \ldots\ [they] inform us only, that they are qualities that effect us in a certain manner \ldots\ as to what they are in themselves, our senses leave us in the dark. \citep[II, 17]{Reid:1969lr}
\end{quote}
(Though one could accept that experience is a way of being affected without denying, as Reid does, that ``secondary qualities'' are maifest in our experience of them.) This conception of phenomenal character is usually associated with either adverbialism \citep[see][]{Ducasse:1942oq,Jackson:1977fk} or belief in \emph{qualia} understood as monadic, nonrepresentational qualities of experience that are immediately present to consciousness \citep[see][]{Block:1996qf,Jackson:1982my}. Shoemaker, however, believes neither in adverbialism nor qualia, so understood, but he evidently shares the more general conception of phenomenal experience as a way of being affected---at least if phenomenal experience is, indeed, what distinguishes these extrinsic causal powers.

This general conception, while commonly held, is not, however, universally held \cite[for criticism see][]{Kalderon:2006fk,Martin:1998nx}. There is an alternative to conceiving of phenomenal experience as a conscious modification of a subject. According to this alternative conception, the phenomenal character of experience is determined by the partial perspective it provides on the chromatic features of the material environment. To know what it is like to undergo a color experience would be to know the color selectively presented to the perceiver's partial perspective \citep[see][166, 172, 173--4]{Nagel:1979fk}. An experience would be necessarily connected to its subject matter since experience, so conceived, just is a perceptual presentation of that subject matter to a perceiver's perspective. 

These distinct conceptions of the phenomenal experience have distinct implications about the causal structure of color perception. 

If the phenomenal character of color experience is understood as a conscious modification of a subject, then the proximate effect of a color's instantiation and viewing is individualistically individuated---as it would be if it were conceived to be, or to be constituted by, the irradiation of a perceiver's sensory surfaces, or more liberally, a pattern of retinal effects. 

If, on the other hand, the phenomenal character of color experience is determined by the presentation of a color to the perceiver's partial perspective, then the proximate effect of a color's instantiation and viewing would not be individualistically individuated. Instead, the proximate effect would be relational---the color's instantiation causes the appropriately situated perceiver to stand in a relation to that color's instantiation. There is nothing incoherent about a cause having a relational effect (where a relational effect is an event constituted by the obtaining of a relation). And there is nothing incoherent about the relational effect of a cause consisting in the obtaining of a relation between a thing and that cause. (Consider the power of the wind to cause a weather vane to point in its direction.)

If the phenomenal character of color experience is determined by the presentation of a color to the perceiver's partial perspective, then the color's instantiation causes the appropriately situated perceiver to stand in a relation to that color's instantiation. Given the phenomenal difference between Norm and Norma's color experience, Norm and Norma stand in different relations to the color's instantiation---each has a unique perspective on the perceived color from which different qualitative aspect's of the color are revealed. (Compare the way in which distinct perspectives can reveal distinct aspects of an object's three-dimensional shape.) But if the proximate effects of a color's instantiation are relational in this way, and Norm and Norma stand in different relations to the color's instantiation, then the color's proximate effects on Norm and Norma themselves differ---which means that distinct causal features of the color are involved in Norm and Norma's perception of that color. If the distinct qualitative aspects of the color are distinct subsets of its causal powers that differ in their proximate effects, then a qualitative aspect of the color must be an \emph{intrinsic} causal power.

Of course, the spectral power distribution of the light reaching the perceiver's eye and its retinal effects, as well as the subsequent, cascading effects of further visual processing at least partly determine the fact that the appropriately situated perceiver stands in the relevant relation to the qualitative aspect of the color presented to his partial perspective. But this does not mean that the proximate effect of a color's instantiation and viewing is the irradiation of the perceiver's sensory surfaces or, more liberally, a pattern of retinal effects. To suppose otherwise would be a kind of level confusion. Compare with the following: Suppose that a subject acquires a belief in light of new evidence. The change in a subject's epistemic state will, of course, be correlated with a change in his neurophysiology and the transition in the subject's neurophysiological states will at least partly determine the transition in the subject's epistemic states. But only an implausible reductionism would maintain that the prior neurophysiological state causes the subsequent epistemic state. On all plausible alternatives, this latter claim exhibits a level confusion \citep[see][]{McDowell:1998lr}. Similarly, if the phenomenal character of color experience is determined by the presentation of a color to the perceiver's partial perspective, then the claim that the proximate effect of a color's instantiation and viewing is the irradiation of the perceiver's sensory surfaces or, more liberally, a pattern of retinal effects exhibits just this kind of level confusion.

Like the Protagorean response considered in the previous section, the present response works, if it works at all, by reducing an apparent case of coincidence to the less metaphysically problematic case of overlap. However, unlike the Protagorean, the present response maintains that the relation between object, perceiver, and circumstance of perception does not determine the color of the object so much as it determines the perceptual availability of a qualitative aspect of that color. The relation determines the subject's perspective on the object's color---a perspective from which the qualitative nature of the color is only partially revealed.

% section sec:intrinsic_and_extrinsic_powers (end)
\section{From Inverted Spectra to Conflicting Appearances}\label{sec:conclusion}

% (fold)

Shoemaker writes of the alleged case of coincidence: 
\begin{quote}
	The inverted spectrum scenario I have described is not the one that has been most frequently discussed in the literature. \citep[270]{Shoemaker:wk} 
\end{quote}
While the usual cases of inverted spectra are behaviorally undetectable, the present inversion ``which involves visual systems that differ somewhat in the relations they `select' to be the relations of color similarity, would of course be behaviorally detectable'' \citep[270]{Shoemaker:wk}. I agree that alleged cases of coincidence are not the usual cases of inverted spectra but only because they are not cases of inverted spectra at all.

Why believe that the alleged cases of coincidence are cases of inverted spectra? The temptation is due to two observations and a misleading conception of the relationship between a color and its qualitative nature. The first observation is this: If different qualitative aspects of a color are presented to the same perceiver in different circumstances of perception or to different perceivers in the same circumstances of perception, then there will be a difference in phenomenal character without a difference in presented color. The second observation is this: If the same qualitative aspect of different colors are presented to the same perceiver in the circumstances of perception or to different perceivers in the same circumstances of perception, then there will be a difference in presented color without a difference in phenomenal character. If the qualitative character of a color is conceived to be a higher-order property---a property of a property, then it is plausible to describe such cases as cases of inverted spectra. While a difference in phenomenal character is explained in terms of a difference in what's present in experience, what makes for the difference is not the color present in experience, but the presence of a distinct property---the qualitative character of the color. On this basis, it is tempting to suppose that the phenomenal character of color experience is \emph{extrinsic} to the presented color. And as we have seen, it is because Shoemaker conceives of the qualitative nature of perceived colors as extrinsic causal powers that he can maintain that there is a necessary correlation between phenomenal character and represented qualitative nature consistent with the possibility of the inverted spectrum.

However, the qualitative nature of a color is not extrinsic to it in the way required for the possibility of the inverted spectrum. I have already argued that the qualitative nature of the color is an intrinsic causal power. Let's, however, approach this matter from another perspective. Perhaps, properly understood, the qualitative character of a color may be conceived as a higher-order property, but this conceptions is incomplete and is thus liable to mislead. The relationship between a color and its qualitative nature is better conceived on the model of the relationship between a determinate and a determinable. A particular shade of red---red$_{17}$, say---has the qualitative character that it does, in part, by being a determinate of the determinable red. In falling under the determinable red, red$_{17}$ bears certain similarity, difference, and exclusion relations to other colors. It is the way that the structure of determinates and determinables in which the colors stand mirrors the similarity, difference, and exclusion relations that obtain among them that makes it apt to conceive of the qualitative character of a color on the model of the relationship between a determinate and a determinable. (All the more so if qualities are thought of as properties of objects that ground comparisons along certain dimensions that admit of degrees.) Red$_{17}$ is a way of being red---it intrinsically is a determinate of the determinable red. Determinates intrinsically are determinations of the determinables under which they stand. That is the respect in which the higher-order property conception is incomplete---it remains silent on the internal relation between a color and its qualitative nature. If, as Shoemaker would have it, the qualitative nature of a color is a subset of its causal powers, they must be intrinsic causal powers. 

The inverted spectrum argument, at least in the context of contemporary philosophy of mind, purports to establish that the phenomenal character of color experience is extrinsic to the presented color. In cases of veridical perceptual variation where the same color is presented in each experience, the distinct qualitative aspects of the presented color are intrinsic to it in a way inconsistent with the possibility of the inverted spectrum, so understood. If perception is partial, as a Heraclitean epistemology would have it, the qualitative character of a color is only ever partially manifest in a given perception. The phenomenal difference is due to the distinct partial perspectives on the perceived color occupied by the same perceiver in different circumstances of perception or by different perceivers in the same circumstance of perception. These distinct partial perspectives reveal different qualitative aspects of the perceived color---qualitative aspects that the color genuinely and intrinsically has. Or consider the case where the distinct colors appear the same in the circumstance of perception. Here, the phenomenal identity is due to the distinct partial perspectives on the different colors occupied by the same perceiver in the circumstance of perception or by different perceivers in the circumstance of perception. These distinct partial perspectives reveal a qualitative aspect shared by distinct perceived colors---a qualitative aspect that the colors genuinely and intrinsically have. The phenomenal character of color experience is not extrinsic to the presented color in the way required for the possibility of the inverted spectrum.

There is a deeper reason why alleged cases of coincidence are not cases of inverted spectrum.

The inverted spectrum hypothesis has been used for a variety of philosophical purposes \citep[see][for some of these]{Byrne:2005ve}. In contemporary philosophy of mind, however, the inverted spectrum hypothesis is used to draw conclusions about the nature of experience---for if the color experiences of two perceivers are spectrally inverted and if the same color is present in each, then it is plausible that the phenomenal character of color experience must be determined by something extrinsic to the presented color. Moreover, it illustrates the perceiver-dependency of phenomenal character. As such, it is one of a battery of considerations that dramatizes the explanatory gap or hard problem of consciousness. For the perceiver-dependency of phenomenal character can encourage the thought that it is constituted by monadic qualities of experience whose instantiation depends on a subject's awareness of them. And it is hard to understand how phenomenal character, so conceived, could be materially realized.

But the inheritance thesis effectively transforms the inverted spectrum argument into the problem of conflicting appearances. The inverted spectrum argument, at least in the context of contemporary philosophy of mind, is an argument about the nature of color experience. In contrast, the problem of conflicting appearances, a much older, indeed, an ancient problem, is a problem about the nature of color. Suppose that the phenomenal character of color experience is inherited from the qualitative nature of the perceived color. If phenomenally distinct color experiences, in a given circumstance of perception, have equal claim to being veridical, then there's a puzzle about the colors presented by these experiences. If Norm perceives an object to be one color and, in the same circumstances of perception, Norma perceives that object, with its color remaining unaltered, to be another color, then what color is the object? Is it one or the other? Or neither? Or both? Notice how in reconciling chromatic inheritance with the possibility of veridical perceptual variation, we were naturally led to speculate about the metaphysics of color. The effect of the inheritance thesis is to transform a problem about color experience into a problem about the colors. Under the influence of the inheritance thesis, the problem of understanding how color perception, given its qualitative character, could be materially realized has become the problem of understanding how the colors, given their perceived qualitative nature, could be materially realized. The mind--body problem, understood as the hard problem of consciousness, has dissolved into the problem of the manifest. \citep[See][for further relevant discussion.]{Byrne:2005jw,Kalderon:2006tg,Sellars:1963eo,Shoemaker:wk} 

The special case of the problem of the manifest that concerns us here is to explain how the colors, given their qualitative nature, can be materially realized. And Price, at least, is confident that this problem can be solved \emph{if} selectionism is true:
\begin{quote}
    It is evident that the Selective Theory is extremely attractive; the more so because it makes the `secondary' qualities no less part of Nature than spatial, temporal, and causal characteristics. Thus it heals the breach between the Nature of the poets and the Nature of the physicists, and perhaps no one but a poet could do full justice to it. \citep[43]{Price:1932fk}
\end{quote}

% (end)
\bibliographystyle{plainnat} 
\bibliography{Philosophy} 

\end{document} 
